\documentclass[fleqn]{MJD}

\usepackage{cancel}
\usepackage{cleveref}
\usepackage{titlesec}
%\colorsections
%\bluelinks
\newcommand{\problem}[1]{\chapter{Problem #1}}
\newcommand{\subproblem}[2]{\section{(#1)~ #2}}
\newcommand{\subsubproblem}[2]{\subsection{ #1)~ #2}}
\newcommand{\U}{\cup}
\renewcommand{\S}{\mathcal{S}}
\renewcommand{\s}{\subset}
\renewcommand{\equiv}{\Leftrightarrow}
\newcommand{\0}{\emptyset}
\newcommand{\imp}{\Rightarrow}
\newcommand{\Usum}{\bigcup\limits_{i=1}^\infty}
\newcommand{\intsum}{\bigcup\limits_{i=1}^\infty}
\newcommand{\infsum}{\sum\limits_{i=1}^\infty}
\newcommand{\sets}{\{A_1, A_2 \dots\} }
\newcommand{\nsets}{\{A_1, \dots, A_n \} }

\titleformat{\chapter}[display]
  {\normalfont\bfseries}{}{0pt}{\LARGE}

%%%%%%%%%%%%%%%%%%%%%%%%%%%%%%%%%%%%
\begin{document}

\titleAT[CS 224N: Assignment 1]{Ryan McMahon}
\large

\begingroup
\let\clearpage\relax
\tableofcontents
\endgroup
\newpage

%-------------------------------------
\problem{1: Softmax (10 pts)}
%-------------------------------------

%----------------------
\subproblem{a}{Softmax Invariance to Constant (5 pts)}
\textit{Prove that softmax is invariant to constant offsets in the input, that is, for any input vector $x$ and any constant $c$, softmax($x$) = softmax($x + c$), where $x + c$ means adding the constant $c$ to every dimension of $x$. Remember that} 
\begin{equation}
	softmax(x)_{i} = \frac{e^{x_{i}}}{\sum_{j} e^{x_{j}}}
\end{equation}


\noindent\textbf{Answer:} \\

\noindent We can show that softmax($x$) = softmax($x+c$) by factoring out $c$ and canceling:

\begin{align}
	softmax(x + c)_{i} &= \frac{e^{x_{i} + c}}{\sum_{j} e^{x_{j} + c}} %
	 					= \frac{e^{x_{i}} \times e^{c}}{e^{c} \times \sum_{j} e^{x_{j}}} \nonumber \\
	%
					   &= \frac{e^{x_{i}} \times \cancel{e^{c}}}{\cancel{e^{c}} \times \sum_{j} e^{x_{j}}} %
					    = softmax(x)_{i} \nonumber
\end{align}

\vskip5em

%----------------------
\subproblem{b}{Softmax Coding (5 pts)}
\textit{Given an input matrix of $N$ rows and $D$ columns, compute the softmax prediction for each
row using the optimization in part (a). Write your implementation in} \verb|q1_softmax.py|. \textit{You may test
by executing} \verb|python q1_softmax.py|. \\

\noindent\textit{Note: The provided tests are not exhaustive. Later parts of the assignment will reference this code so it is important to have a correct implementation. Your implementation should also be efficient and vectorized whenever possible (i.e., use numpy matrix operations rather than for loops). A non-vectorized implementation will not receive full credit!} \\

\noindent \textbf{Answer:} \\

\noindent See code: $\sim$\verb|/code/q1_softmax.py|.



\newpage
%-------------------------------------
\problem{2: Neural Network Basics (30 pts)}
%-------------------------------------


%----------------------
\subproblem{a}{Sigmoid Gradient (3 pts)}

\textit{Derive the gradients of the sigmoid function and show that it can be rewritten as a function of the function value (i.e., in some expression where only σ(x), but not x, is present). Assume that the input x is a scalar for this question. Recall, the sigmoid function is}

\begin{equation}
	\sigma(x) = \frac{1}{1 + e^{-x}}
\end{equation}


\noindent \textbf{Answer:}

\begin{align}
	\sigma(x) &= \frac{1}{1 + e^{-x}} \nonumber \\
	%
			  &= \frac{e^{x}}{1 + e^{x}} \nonumber \\
	%
	\frac{\partial}{\partial x} \sigma(x) &= \frac{e^{x} \times (1 + e^{x}) - (e^{x} \times e^{x})}{(1 + e^{x})^{2}} \nonumber \\
	%
			  &= \frac{e^{x} + \cancel{(e^{x} \times e^{x})} - \cancel{(e^{x} \times e^{x})}}{(1 + e^{x})^{2}} \nonumber \\
	%
			  &= \frac{e^{x}}{(1 + e^{x})^{2}}  = \sigma(x) \times (1 - \sigma(x)) \nonumber 
\end{align}

\vskip2em
\noindent Because $1 - \sigma(x) = \sigma(-x)$ we can show that:

\begin{align}
	\frac{\partial}{\partial x} \sigma(x) &= \frac{e^{x}}{(1 + e^{x})^{2}} \nonumber \\
	%
			  &= \sigma(x) \times \sigma(-x) \nonumber \\
	%
			  &= \frac{e^{x}}{1 + e^{x}} \times \frac{1}{1 + e^{+x}} \nonumber \\
	%
			  &= \frac{e^{x}}{(1 + e^{x})^{2}} \nonumber
\end{align}



%----------------------
\newpage
\subproblem{b}{Softmax Gradient w/ Cross Entropy Loss (3 pts)}
\label{prob:2b}

\textit{Derive the gradient with regard to the inputs of a softmax function when cross entropy loss is used for evaluation, i.e., find the gradients with respect to the softmax input vector $\bm{\theta}$, when the prediction is made by $\hat{\mathbf{y}} = softmax(\bm{\theta})$. Remember the cross entropy function is}

\begin{equation}
	CE(\mathbf{y},\hat{\mathbf{y}}) = - \sum_{i} y_{i} \times log(\hat{y_{i}})
\end{equation}

\noindent \textit{where $\mathbf{y}$ is the one-hot label vector, and $\hat{\mathbf{y}}$ is the predicted probability vector for all classes. (Hint: you might want to consider the fact many elements of $\mathbf{y}$ are zeros, and assume that only the $k-th$ dimension
of $\mathbf{y}$ is one.)} \\

\noindent \textbf{Answer:} \\

\noindent Let $S$ represent the softmax function:

\begin{align}
	f_{i} &= e^{\theta_{i}} \nonumber \\
	%
	g_{i} &= \sum_{k=1}^{K} e^{\theta_{k}} \nonumber \\
	%
	S_{i} &= \frac{f_{i}}{g_{i}} \nonumber \\
	%
	\frac{\partial S_{i}}{\partial \theta_{j}} &= \frac{f'_{i} g_{i} - g'_{i} f_{i}}{g_{i}^{2}} \nonumber
\end{align}

\vskip2em
%
\noindent So if $i = j$:
\begin{align}
	f'_{i} &= f_{i}; \hspace*{5pt} g'_{i} = e^{\theta_{j}} \nonumber \\
	%
	\frac{\partial S_{i}}{\partial \theta_{j}} &= \frac{e^{\theta_{i}} \sum_{k} e^{\theta_{k}} - e^{\theta_{j}} e^{\theta_{i}} }{ (\sum_{k} e^{\theta_{k}})^{2} } \nonumber \\
	%
		   &= \frac{e^{\theta_{i}}}{\sum_{k} e^{\theta_{k}}} \times \frac{\sum_{k} e^{\theta_{k}} - e^{\theta_{j}}}{\sum_{k} e^{\theta_{k}}} \nonumber \\
	%
		   &= S_{i} \times (1 - S_{i}) \nonumber
\end{align}

\vskip2em
%
\noindent And if $i \ne j$:
\begin{align}
	\frac{\partial S_{i}}{\partial \theta_{j}} %
		&= \frac{0 - e^{\theta_{j}} e^{\theta_{i}}}{(\sum_{k} e^{\theta_{k}})^{2}} \nonumber \\
	%
		&= - \frac{e^{\theta_{j}}}{\sum_{k} e^{\theta_{k}}} \times \frac{e^{\theta_{i}}}{\sum_{k} e^{\theta_{k}}} \nonumber \\
	%
		&= -S_{j} \times S_{i} \nonumber
\end{align}

\newpage

\noindent We can now use these when operating on our loss function (let $L$ represent the cross entropy function):
\begin{align}
	\frac{\partial L}{\partial \theta_{i}} % 
		&= - \sum_{k} y_{k} \frac{\partial log S_{k}}{\partial \theta_{i}}	\nonumber \\
	%
		&= - \sum_{k} y_{k} \frac{1}{S_{k}} \frac{\partial S_{k}}{\partial \theta_{i}} \nonumber \\
	%
		&= - y_{i} (1 - S_{i}) - \sum_{k \ne i} y_{k} \frac{1}{S_{k}} (-S_{k} \times S_{i}) \nonumber \\
	% 
		&= - y_{i} (1 - S_{i}) + \sum_{k \ne i} y_{k} S_{i} \nonumber \\
	%
		&= - y_{i} + y_{i} S_{i} + \sum_{k \ne i} y_{k} S_{i} \nonumber \\
	%
		&= S_{i}(\sum_{k} y_{k}) - y_{i} \nonumber
\end{align}

\vskip2em

\noindent And because we know that $\sum_{k} y_{k} = 1$:
\begin{align}
	\frac{\partial L}{\partial \theta_{i}} % 
			&= S_{i} - y_{i} \nonumber
\end{align} 



%----------------------
\newpage
\subproblem{c}{One Hidden Layer Gradient (6 pts)}

\textit{Derive the gradients with respect to the inputs $\bm{x}$ to a one-hidden-layer neural network (that is, find $\frac{\partial J}{\partial \bm{x}}$ where $J = CE(\mathbf{y}, \hat{\mathbf{y}})$ is the cost function for the neural network). The neural network employs sigmoid activation function for the hidden layer, and softmax for the output layer. Assume the one-hot label vector is $\mathbf{y}$, and cross entropy cost is used. (Feel free to use $\sigma'(x)$ as the shorthand for sigmoid gradient, and feel free to define any variables whenever you see fit.)} \\

\noindent \textit{Recall that forward propoagation is as follows}
\begin{align}
	\mathbf{h} &= sigmoid(\bm{xW}_{1} + \bm{b}_{1})	& \hat{\bm{y}} = softmax(\bm{hW}_{2} + \bm{b}_{2}) \nonumber
\end{align}

\vskip2em

\noindent \textbf{Answer:} \\

\noindent Let $f_{2} = \bm{xW}_{1} + \bm{b}_{1}$ and $f_{3} = \bm{hW}_{2} + \bm{b}_{2}$;

\begin{align}
	\frac{\partial J}{\partial f_{3}} %
		&= \bm{\delta}_{3} = \hat{\bm{y}} - \bm{y} \nonumber \\
	%
	\frac{\partial J}{\partial \bm{h}} %
		&= \bm{\delta}_{2} = \bm{\delta}_{3}\bm{W}_{2}^{T} \nonumber \\
	%
	\frac{\partial J}{\partial f_{2}} %
			&= \bm{\delta}_{1} = \bm{\delta}_{2} \circ \sigma'(f_{2}) \nonumber \\
	%
	\frac{\partial J}{\partial \bm{x}} %
		&= \bm{\delta}_{1} \frac{\partial f_{2}}{\partial \bm{x}} \nonumber \\
	%
		&= \bm{\delta}_{1}  \bm{W}_{1}^{T} \nonumber
\end{align}

\vskip5em

%----------------------
\subproblem{d}{No. Parameters (2 pts)}

\textit{How many parameters are there in this neural network} [from (\textbf{c}) above], \textit{assuming the input is $D_{x}-$dimensional, the output is $D_{y}-$dimensional, and there are $H$ hidden units?} \\

\noindent \textbf{Answer:} \\

\begin{align}
	n_{W_{1}} &= D_{x} \times H \nonumber \\
	%
	n_{b_{1}} &= H \nonumber \\
	%
	n_{W_{2}} &= H \times D_{y} \nonumber \\
	%
	n_{b_{2}} &= D_{y} \nonumber \\
	%
	N &= (D_{x} \times H) + H + (H \times D_{y}) + D_{y} \nonumber
\end{align}



%----------------------
\newpage
\subproblem{e}{Sigmoid Activation Code (4 pts)}

\textit{Fill in the implementation for the sigmoid activation function and its gradient in} \verb|q2_sigmoid.py|. \textit{Test your implementation using} \verb|python q2_sigmoid.py|. \textit{Again, thoroughly test your code as the provided tests may not be exhaustive.} \\

\noindent \textbf{Answer:} \\

\noindent See code: $\sim$\verb|/code/q2_sigmoid.py|.

\vskip5em


%----------------------
\subproblem{f}{Gradient Check Code (4 pts)}

\textit{To make debugging easier, we will now implement a gradient checker. Fill in the implementation for} \verb|gradcheck_naive| \textit{in} \verb|q2_gradcheck.py|. \textit{Test your code using} \verb|python q2_gradcheck.py|. \\

\noindent \textbf{Answer:} \\

\noindent See code: $\sim$\verb|/code/q2_gradcheck.py|.

\vskip5em


%----------------------
\subproblem{g}{Neural Net Code (8 pts)}

\textit{Now, implement the forward and backward passes for a neural network with one sigmoid hidden layer. Fill in your implementation in} \verb|q2_neural.py|. \textit{Sanity check your implementation with} \verb|python q2_neural.py|. \\

\noindent \textbf{Answer:} \\

\noindent See code: $\sim$\verb|/code/q2_neural.py|.



\newpage

%-------------------------------------
\problem{3: Word2Vec (40 pts + 2 bonus)}
%-------------------------------------

%----------------------
\subproblem{a}{Context Word Gradients (3 pts)}

\textit{Assume you are given a predicted word vector $\bm{v}_{c}$ corresponding to the center word $\bm{c}$ for skipgram, and word prediction is made with the softmax function found in word2vec models}

\begin{align}
	\hat{\bm{y}}_{o} &= p(\bm{o} \hspace*{1pt} \vert \hspace*{1pt} \bm{c}) = %
					\frac{exp(\bm{u}_{0}^{T} \bm{v}_{c})}{\sum\limits_{w=1}^{W} exp(\bm{u}_{w}^{T} \bm{v}_{c})}
\end{align}

\noindent \textit{where $\bm{w}$ denotes the w-th word and $\bm{u}_{w}$ ($w = 1,...,W$) are the ``output'' word vectors for all words in the vocabulary. Assume cross entropy cost is applied to this prediction and word $\bm{o}$ is the expected word (the $\bm{o}$-th element of the one-hot label vector is one), derive the gradients with respect to $\bm{v}_c$, } \\

Hint: It will be helpful to use notation from question 2. For instance, letting $\hat{\bm{y}}$ be the vector of softmax predictions for every word, $\bm{y}$ as the expected word vector, and the loss function

\begin{align}
	J_{softmax-CE}(\bm{o}, \bm{v}_{c}, \bm{U}) &= CE(\bm{y}, \hat{\bm{y}})
\end{align}

\noindent \textit{where $\bm{U}$ = [$\bm{u}_{1}$,$\bm{u}_{1}$,$\dots$, $\bm{u}_{W}$] is the matrix of all the output vectors. Make sure you state the orientation of your vectors and matrices.} \\


\noindent \textbf{Answer:} \\

\noindent From Problem~\ref{prob:2b} we know that $\frac{\partial J}{\partial \bm{\theta}} = (\hat{\bm{y}} - \bm{y})$. Given that, let $\bm{theta} = \bm{v}_{c}$. Then 

\begin{align}
	\frac{\partial J}{\partial \bm{\theta}} &= %
	 		\bm{U}^{T} (\hat{\bm{y}} - \bm{y}) \nonumber
\end{align}



\newpage
%----------------------
\subproblem{b}{Output Word Gradients (3 pts)}

\textit{As in the previous part, derive gradients for the “output” word vectors $\bm{u}_{w}$ 's (including $\bm{u}_{o}$).} \\

\noindent \textbf{Answer:} \\

\noindent Here we're going to do essentially the same thing, but instead transpose the error. So, from above and Problem~\ref{prob:2b}, let $\bm{\theta} = \bm{U}$


\begin{align}
	\frac{\partial J}{\partial \bm{\theta}} &= %
		 		\bm{v}_{c} (\hat{\bm{y}} - \bm{y})^{T} \nonumber
\end{align}

\vskip5em



%----------------------
\subproblem{c}{Repeat Gradients with Negative Sampling Loss (6 pts)}

\textit{Repeat part \textbf{(a)} and \textbf{(b)} assuming we are using the negative sampling loss for the predicted vector $\bm{v}_{c}$, and the expected output word is $\bm{o}$. Assume that $\bm{K}$ negative samples (words) are drawn, and they are $\bm{1},\dots ,\bm{K}$, respectively for simplicity of notation ($o \notin \{1,...,K\}$). Again, for a given word, $o$,denote its output vector as $\bm{u}_{o}$. The negative sampling loss function in this case is}

\begin{align}
	J_{neg-sample}(\bm{o}, \bm{v}_{c}, \bm{U}) &= %
		− log(\sigma(\bm{u}_{o}^{T} \bm{v}_{c})) - \sum\limits_{k=1}^{K} log(\sigma(-\bm{u}_{k}^{T} \bm{v}_{c}))
\end{align}

\noindent where $\sigma(\cdot)$ is the sigmoid function. \\

\noindent \textit{After you’ve done this, describe with one sentence why this cost function is much more efficient to compute than the softmax-CE loss (you could provide a speed-up ratio, i.e. the runtime of the softmax-CE loss divided by the runtime of the negative sampling loss).} \\


\noindent \textbf{Answer:} \\

\noindent Let $z_{j} = \bm{u}_{j}^{T} \bm{v}_{c}$:

\begin{align}
	\frac{\partial J}{\partial z_{j}} &= %
		\left\{
			\begin{array}{ll}
				 \sigma(\bm{u}_{j}^{T} \bm{v}_{c}) - 1 & \mbox{if } j = o \\ 
				 \sigma(\bm{u}_{j}^{T} \bm{v}_{c}) 	   & \mbox{if } j \in \bm{K}
			\end{array}
		\right. \nonumber 
\end{align}

\noindent Then we can separate out the partials for $\bm{u}_{j}$ and $\bm{v}_{c}$.

\begin{align}
	\frac{\partial J}{\partial \bm{u}_{o}} &= %
		\frac{\partial J}{\partial z_{j}} \times %
			\frac{ \partial z_{j} }{ \partial \bm{u}_{o} } \nonumber \\
		%
		&= (\sigma(\bm{u}_{o}^{T} \bm{v}_{c}) - 1) \bm{v}_{c} \nonumber \\
	%
	\frac{\partial J}{\partial \bm{u}_{k}} &= %
			\frac{\partial J}{\partial z_{j}} \times %
				\frac{ z_{j} }{ \partial \bm{u}_{k} } \nonumber \\
		%
		&= -(\sigma(-\bm{u}_{k}^{T} \bm{v}_{c}) - 1) \bm{v}_{c} \mbox{  for all } k \in \bm{K} \nonumber
\end{align}

\begin{align}
	\frac{\partial J}{\partial \bm{v}_{c}} &= %
		\frac{\partial J}{\partial z_{j}} \times %
			\frac{ \partial z_{j}}{ \partial \bm{v}_{c} } \nonumber \\
		%
		&= (\sigma(\bm{u}_{o}^{T} \bm{v}_{c}) -1) \bm{u}_{o} - %
			\sum\limits_{k=1}^{K} (\sigma(- \bm{u}_{k}^{T} \bm{v}_{c}) - 1) \bm{u}_{k} \nonumber
\end{align}


\noindent This is faster than the original cross entropy loss because we are no longer deriving the gradients $\forall w_{j} \in W$. Instead, we are only evaluating the gradients for $[w_{o}, w_{k},\dots,w_{K}]$. 


\vskip5em


%----------------------
\subproblem{d}{Skip-gram and CBOW Gradients (8 pts)}
\textit{Derive gradients for all of the word vectors for skip-gram and CBOW given the previous parts and given a set of context words [word$_{c−m}$ ,...,word$_{c−1}$ ,word$_{c}$ ,word$_{c+1}$ ,...,word$_{c+m}$ ], where $m$ is the context size. Denote the “input” and “output” word vectors for word $k$ as $v_{k}$ and $u_{k}$ respectively.} \\

\noindent \textit{Hint: Feel free to use $F(\bm{o}, \bm{v}_{c})$ (where $\bm{o}$ is the expected word) as a placeholder for the $J_{softmax-CE}(\bm{o},\bm{v}_{c},\dots)$ or $J_{neg-sample}(\bm{o},\bm{v}_{c},\dots)$ cost functions in this part -- you'll see that this is a useful abstraction for the coding part. That is, your solution may contain terms of the form $\frac{\partial F(\bm{o}, \bm{v}_{c})}{\partial \dots}$.} \\
%
\noindent \textit{Recall that for skip-gram, the cost for a context centered around $c$ is }
%
\begin{align}
	J_{skip-gram}(word_{c-m \dots c+m}) &= %
		\sum\limits_{-m \leq j \leq m, j \ne 0} F(\bm{w}_{c+j}, \bm{v}_{c}) 
\end{align}
\noindent \textit{where $\bm{w}_{c+j}$ refers to the word at the $j-$th index from the center.} \\

\noindent \textit{CBOW is slightly different. Instead of using $\bm{v}_{c}$ as the predicted vector, we use $\hat{\bm{v}}$ defined below. For (a simpler variant of) CBOW, we sum up the input word vectors in the context}
%
\begin{align}
	\hat{\bm{v}} &= \sum\limits_{-m \leq j \leq m, j \ne 0} \bm{v}_{c+j}
\end{align}
\noindent then the CBOW cost is
%
\begin{align}
	J_{CBOW}(word_{c-m \dots c+m}) &= F(\bm{w}_{c}, \hat{\bm{v}})
\end{align}

\noindent \textit{Note: To be consistent with the $\hat{\bm{v}}$ notation such as for the code portion, for skip-gram $\hat{\bm{v}} = \bm{v}_{c}$.} \\


\noindent \textbf{Answer:} \\

\noindent For the skip-gram model, we can show that (for a given context window) the gradients are equal to:
%
\begin{align}
	\frac{J_{skip-gram}(word_{c-m \dots c+m})}{\partial \bm{U}} &= %
		\sum\limits_{-m \leq j \leq m, j \ne 0} \frac{\partial F(\bm{w}_{c+j}, \bm{v}_{c})}{\partial \bm{U}} \nonumber \\
	%
	\frac{J_{skip-gram}(word_{c-m \dots c+m})}{\partial \bm{v}_{c}} &= %
		\sum\limits_{-m \leq j \leq m, j \ne 0} \frac{\partial F(\bm{w}_{c+j}, \bm{v}_{c})}{\partial \bm{v}_{c}} \nonumber \\
	%
	\frac{J_{skip-gram}(word_{c-m \dots c+m})}{\partial \bm{v}_{j}} &= 0, \forall j\ne c \nonumber
\end{align}


\noindent For the CBOW model, alternatively, the gradients are equal to:
%
\begin{align}
	\frac{J_{CBOW}(word_{c-m \dots c+m})}{\partial \bm{U}} &= %
		\frac{\partial F(\bm{w}_{c}, \hat{\bm{v}})}{\partial \bm{U}} \nonumber \\
	%
	\frac{J_{CBOW}(word_{c-m \dots c+m})}{\partial \bm{v}_{j}} &= %
		\frac{\partial F(\bm{w}_{c}, \hat{\bm{v}})}{\partial \hat{\bm{v}}}, \forall (j \ne c) \in \{c-m \dots c+m\}  \nonumber \\
	%
	\frac{J_{CBOW}(word_{c-m \dots c+m})}{\partial \bm{v}_{j}} &= 0, %
		\forall (j \ne c) \notin \{c-m \dots c+m\}  \nonumber
\end{align}
%
\noindent The qualifying indexes for the latter two equations are essentially saying ``where $j$ is in the range $\{c-m \dots c+m\}$, if the middle term ($c-0$) is removed''; and then the same thing, but ``where $j$ is not in that range''.


\vskip5em



%----------------------
\subproblem{e}{Word2Vec with SGD Code (12 pts)}
\textit{In this part you will implement the word2vec models and train your own word vectors
with stochastic gradient descent (SGD). First, write a helper function to normalize rows of a matrix in}
\verb|q3_word2vec.py|. \textit{In the same file, fill in the implementation for the softmax and negative sampling cost and gradient functions. Then, fill in the implementation of the cost and gradient functions for the skip-gram model. When you are done, test your implementation by running} \verb|python q3_word2vec.py|.
\textit{Note: If you choose not to implement CBOW (part h), simply remove the} \verb|NotImplementedError| \textit{so that your tests will complete.} \\


\noindent \textbf{Answer:} \\

\noindent See code: $\sim$\verb|/code/q3_word2vec.py|.




\end{document}
